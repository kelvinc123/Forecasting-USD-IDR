\documentclass[]{article}
\usepackage{lmodern}
\usepackage{amssymb,amsmath}
\usepackage{ifxetex,ifluatex}
\usepackage{fixltx2e} % provides \textsubscript
\ifnum 0\ifxetex 1\fi\ifluatex 1\fi=0 % if pdftex
  \usepackage[T1]{fontenc}
  \usepackage[utf8]{inputenc}
\else % if luatex or xelatex
  \ifxetex
    \usepackage{mathspec}
  \else
    \usepackage{fontspec}
  \fi
  \defaultfontfeatures{Ligatures=TeX,Scale=MatchLowercase}
\fi
% use upquote if available, for straight quotes in verbatim environments
\IfFileExists{upquote.sty}{\usepackage{upquote}}{}
% use microtype if available
\IfFileExists{microtype.sty}{%
\usepackage{microtype}
\UseMicrotypeSet[protrusion]{basicmath} % disable protrusion for tt fonts
}{}
\usepackage[margin=1in]{geometry}
\usepackage{hyperref}
\hypersetup{unicode=true,
            pdftitle={Forecasting USD/IDR exchange rate},
            pdfauthor={Kelvin Christian, Kristoffer Chandra, Vincent Widiaman},
            pdfborder={0 0 0},
            breaklinks=true}
\urlstyle{same}  % don't use monospace font for urls
\usepackage{color}
\usepackage{fancyvrb}
\newcommand{\VerbBar}{|}
\newcommand{\VERB}{\Verb[commandchars=\\\{\}]}
\DefineVerbatimEnvironment{Highlighting}{Verbatim}{commandchars=\\\{\}}
% Add ',fontsize=\small' for more characters per line
\usepackage{framed}
\definecolor{shadecolor}{RGB}{248,248,248}
\newenvironment{Shaded}{\begin{snugshade}}{\end{snugshade}}
\newcommand{\KeywordTok}[1]{\textcolor[rgb]{0.13,0.29,0.53}{\textbf{#1}}}
\newcommand{\DataTypeTok}[1]{\textcolor[rgb]{0.13,0.29,0.53}{#1}}
\newcommand{\DecValTok}[1]{\textcolor[rgb]{0.00,0.00,0.81}{#1}}
\newcommand{\BaseNTok}[1]{\textcolor[rgb]{0.00,0.00,0.81}{#1}}
\newcommand{\FloatTok}[1]{\textcolor[rgb]{0.00,0.00,0.81}{#1}}
\newcommand{\ConstantTok}[1]{\textcolor[rgb]{0.00,0.00,0.00}{#1}}
\newcommand{\CharTok}[1]{\textcolor[rgb]{0.31,0.60,0.02}{#1}}
\newcommand{\SpecialCharTok}[1]{\textcolor[rgb]{0.00,0.00,0.00}{#1}}
\newcommand{\StringTok}[1]{\textcolor[rgb]{0.31,0.60,0.02}{#1}}
\newcommand{\VerbatimStringTok}[1]{\textcolor[rgb]{0.31,0.60,0.02}{#1}}
\newcommand{\SpecialStringTok}[1]{\textcolor[rgb]{0.31,0.60,0.02}{#1}}
\newcommand{\ImportTok}[1]{#1}
\newcommand{\CommentTok}[1]{\textcolor[rgb]{0.56,0.35,0.01}{\textit{#1}}}
\newcommand{\DocumentationTok}[1]{\textcolor[rgb]{0.56,0.35,0.01}{\textbf{\textit{#1}}}}
\newcommand{\AnnotationTok}[1]{\textcolor[rgb]{0.56,0.35,0.01}{\textbf{\textit{#1}}}}
\newcommand{\CommentVarTok}[1]{\textcolor[rgb]{0.56,0.35,0.01}{\textbf{\textit{#1}}}}
\newcommand{\OtherTok}[1]{\textcolor[rgb]{0.56,0.35,0.01}{#1}}
\newcommand{\FunctionTok}[1]{\textcolor[rgb]{0.00,0.00,0.00}{#1}}
\newcommand{\VariableTok}[1]{\textcolor[rgb]{0.00,0.00,0.00}{#1}}
\newcommand{\ControlFlowTok}[1]{\textcolor[rgb]{0.13,0.29,0.53}{\textbf{#1}}}
\newcommand{\OperatorTok}[1]{\textcolor[rgb]{0.81,0.36,0.00}{\textbf{#1}}}
\newcommand{\BuiltInTok}[1]{#1}
\newcommand{\ExtensionTok}[1]{#1}
\newcommand{\PreprocessorTok}[1]{\textcolor[rgb]{0.56,0.35,0.01}{\textit{#1}}}
\newcommand{\AttributeTok}[1]{\textcolor[rgb]{0.77,0.63,0.00}{#1}}
\newcommand{\RegionMarkerTok}[1]{#1}
\newcommand{\InformationTok}[1]{\textcolor[rgb]{0.56,0.35,0.01}{\textbf{\textit{#1}}}}
\newcommand{\WarningTok}[1]{\textcolor[rgb]{0.56,0.35,0.01}{\textbf{\textit{#1}}}}
\newcommand{\AlertTok}[1]{\textcolor[rgb]{0.94,0.16,0.16}{#1}}
\newcommand{\ErrorTok}[1]{\textcolor[rgb]{0.64,0.00,0.00}{\textbf{#1}}}
\newcommand{\NormalTok}[1]{#1}
\usepackage{graphicx,grffile}
\makeatletter
\def\maxwidth{\ifdim\Gin@nat@width>\linewidth\linewidth\else\Gin@nat@width\fi}
\def\maxheight{\ifdim\Gin@nat@height>\textheight\textheight\else\Gin@nat@height\fi}
\makeatother
% Scale images if necessary, so that they will not overflow the page
% margins by default, and it is still possible to overwrite the defaults
% using explicit options in \includegraphics[width, height, ...]{}
\setkeys{Gin}{width=\maxwidth,height=\maxheight,keepaspectratio}
\IfFileExists{parskip.sty}{%
\usepackage{parskip}
}{% else
\setlength{\parindent}{0pt}
\setlength{\parskip}{6pt plus 2pt minus 1pt}
}
\setlength{\emergencystretch}{3em}  % prevent overfull lines
\providecommand{\tightlist}{%
  \setlength{\itemsep}{0pt}\setlength{\parskip}{0pt}}
\setcounter{secnumdepth}{0}
% Redefines (sub)paragraphs to behave more like sections
\ifx\paragraph\undefined\else
\let\oldparagraph\paragraph
\renewcommand{\paragraph}[1]{\oldparagraph{#1}\mbox{}}
\fi
\ifx\subparagraph\undefined\else
\let\oldsubparagraph\subparagraph
\renewcommand{\subparagraph}[1]{\oldsubparagraph{#1}\mbox{}}
\fi

%%% Use protect on footnotes to avoid problems with footnotes in titles
\let\rmarkdownfootnote\footnote%
\def\footnote{\protect\rmarkdownfootnote}

%%% Change title format to be more compact
\usepackage{titling}

% Create subtitle command for use in maketitle
\providecommand{\subtitle}[1]{
  \posttitle{
    \begin{center}\large#1\end{center}
    }
}

\setlength{\droptitle}{-2em}

  \title{Forecasting USD/IDR exchange rate}
    \pretitle{\vspace{\droptitle}\centering\huge}
  \posttitle{\par}
    \author{Kelvin Christian, Kristoffer Chandra, Vincent Widiaman}
    \preauthor{\centering\large\emph}
  \postauthor{\par}
      \predate{\centering\large\emph}
  \postdate{\par}
    \date{May 28, 2019}


\begin{document}
\maketitle

\subsection{I. Introduction}\label{i.-introduction}

The data we choose to analyze is the currency exchange of US Dollar with
Indonesian Rupiah (Indonesian Rupiah per 1 US dollar). We will also use
the data of S\&P 500 as the gauge to measure the development of the U.S.
economy. The reason we choose S\&P 500 is because it can represent the
condition of the U.S. economy. Our hypothesis says that the increase in
the dollar price in rupiah is moderately correlated with the growth of
S\&P 500 index.

\subsection{II. Results}\label{ii.-results}

Import the library for forecasting

\begin{Shaded}
\begin{Highlighting}[]
\KeywordTok{library}\NormalTok{(quantmod)}
\KeywordTok{library}\NormalTok{(ggplot2)}
\KeywordTok{library}\NormalTok{(forecast)}
\KeywordTok{library}\NormalTok{(vars)}
\KeywordTok{library}\NormalTok{(lmtest)}
\KeywordTok{library}\NormalTok{(timeSeries)}
\KeywordTok{library}\NormalTok{(rugarch)}
\KeywordTok{library}\NormalTok{(tseries)}
\KeywordTok{library}\NormalTok{(Metrics)}
\KeywordTok{library}\NormalTok{(strucchange)}
\end{Highlighting}
\end{Shaded}

\subsubsection{a. Time Series plot}\label{a.-time-series-plot}

Using getSymbols function to download the daily data from the Yahoo
finance. The data that we'll be analyzing is the monthly return data.
So, we need to first convert from daily prices to monthly return time
series, and calculate the return.

\begin{Shaded}
\begin{Highlighting}[]
\CommentTok{#Get necessary data. We will convert the data to monthly data for compatibility}
\NormalTok{usd_idr <-}\StringTok{ }\KeywordTok{getSymbols}\NormalTok{(}\StringTok{"IDR=X"}\NormalTok{,}\DataTypeTok{auto.assign =} \OtherTok{FALSE}\NormalTok{, }\DataTypeTok{from =} \KeywordTok{as.Date}\NormalTok{(}\StringTok{"01/01/2002"}\NormalTok{, }\DataTypeTok{format =} \StringTok{"%m/%d/%Y"}\NormalTok{), }\DataTypeTok{to =} \KeywordTok{as.Date}\NormalTok{(}\StringTok{"05/28/2019"}\NormalTok{, }\DataTypeTok{format =} \StringTok{"%m/%d/%Y"}\NormalTok{))[,}\DecValTok{6}\NormalTok{]}
\NormalTok{usd_idr <-}\StringTok{ }\KeywordTok{to.monthly}\NormalTok{(usd_idr)[,}\DecValTok{4}\NormalTok{]}

\NormalTok{sp500 <-}\StringTok{ }\KeywordTok{getSymbols}\NormalTok{(}\StringTok{"^GSPC"}\NormalTok{,}\DataTypeTok{auto.assign =} \OtherTok{FALSE}\NormalTok{, }\DataTypeTok{from =} \KeywordTok{as.Date}\NormalTok{(}\StringTok{"01/01/2002"}\NormalTok{, }\DataTypeTok{format =} \StringTok{"%m/%d/%Y"}\NormalTok{), }\DataTypeTok{to =} \KeywordTok{as.Date}\NormalTok{(}\StringTok{"05/28/2019"}\NormalTok{, }\DataTypeTok{format =} \StringTok{"%m/%d/%Y"}\NormalTok{))[,}\DecValTok{6}\NormalTok{]}
\NormalTok{sp500 <-}\StringTok{ }\KeywordTok{to.monthly}\NormalTok{(sp500)[,}\DecValTok{4}\NormalTok{]}

\CommentTok{#Convert Data to time series and calculate returns}
\NormalTok{usd_idr_ts <-}\StringTok{ }\KeywordTok{ts}\NormalTok{(usd_idr, }\DataTypeTok{start =} \DecValTok{2002}\NormalTok{, }\DataTypeTok{freq =} \DecValTok{12}\NormalTok{)}
\NormalTok{sp500_ts <-}\StringTok{ }\KeywordTok{ts}\NormalTok{(sp500, }\DataTypeTok{start =} \DecValTok{2002}\NormalTok{, }\DataTypeTok{freq =} \DecValTok{12}\NormalTok{)}

\NormalTok{usd_idr_ret=}\DecValTok{100}\OperatorTok{*}\NormalTok{(}\KeywordTok{diff}\NormalTok{(usd_idr_ts)}\OperatorTok{/}\KeywordTok{lag}\NormalTok{(usd_idr_ts, }\OperatorTok{-}\DecValTok{1}\NormalTok{))}
\NormalTok{sp500_ret=}\DecValTok{100}\OperatorTok{*}\NormalTok{(}\KeywordTok{diff}\NormalTok{(sp500_ts)}\OperatorTok{/}\KeywordTok{lag}\NormalTok{(sp500_ts, }\OperatorTok{-}\DecValTok{1}\NormalTok{))}
\end{Highlighting}
\end{Shaded}

Plot the data to get a picture of the return movement

\begin{Shaded}
\begin{Highlighting}[]
\CommentTok{# plot for USD/IDR exchange rate (monthly)}
\KeywordTok{plot}\NormalTok{(usd_idr_ret, }\DataTypeTok{main =} \StringTok{"Monthly return of dollars in Indonesian rupiah (IDR)"}\NormalTok{) }
\end{Highlighting}
\end{Shaded}

\includegraphics{Project_2_files/figure-latex/plot-1.pdf}

\begin{Shaded}
\begin{Highlighting}[]
\CommentTok{# plot for S&P500 (monthly)}
\KeywordTok{plot}\NormalTok{(sp500_ret, }\DataTypeTok{main =} \StringTok{"S&P500 monthly return"}\NormalTok{) }
\end{Highlighting}
\end{Shaded}

\includegraphics{Project_2_files/figure-latex/plot-2.pdf}

Since we are analyzing combined data, it is reasonable to plot both time
series in one plot. In order to do that, we need to standardize the
return first so they could fit together.

\begin{Shaded}
\begin{Highlighting}[]
\KeywordTok{plot}\NormalTok{((usd_idr_ret}\OperatorTok{-}\KeywordTok{mean}\NormalTok{(usd_idr_ret))}\OperatorTok{/}\KeywordTok{sd}\NormalTok{(usd_idr_ret), }\DataTypeTok{main =} \StringTok{"USD/IDR vs. S&P 500 returns"}\NormalTok{, }\DataTypeTok{ylab =} \StringTok{"Returns"}\NormalTok{)}
\KeywordTok{lines}\NormalTok{((sp500_ret}\OperatorTok{-}\KeywordTok{mean}\NormalTok{(sp500_ret))}\OperatorTok{/}\KeywordTok{sd}\NormalTok{(sp500_ret), }\DataTypeTok{col =} \StringTok{"red"}\NormalTok{)}
\end{Highlighting}
\end{Shaded}

\includegraphics{Project_2_files/figure-latex/combined plot-1.pdf}

Looking at this combined plot, we can see that at year 2009, the returns
for S\&P 500 spikes down while the returns for USD/IDR spikes up. We
initially predicted that the return of S\&P 500 may explain the return
of USD/IDR in a direct relationship. But this plot contradicts our
initial assumption because the returns of dollar price spikes up when
the returns of S\&P 500 spikes down. They seem to have an inverse
relationship.

Then, we now consider the plot of ACF and PACF for both USD/IDR and S\&P
500 returns to do further analysis and forecast.

\begin{Shaded}
\begin{Highlighting}[]
\KeywordTok{tsdisplay}\NormalTok{(usd_idr_ret, }\DataTypeTok{main =} \StringTok{"USD/IDR Returns"}\NormalTok{)}
\end{Highlighting}
\end{Shaded}

\includegraphics{Project_2_files/figure-latex/tsdisplay usd/idr-1.pdf}

Even though it looks like white noise, we can see that the ACF and PACF
for the USD/IDR spikes at lag 2 while spikes at the other lags are not
different from 0. Thus, it is reasonable to consider ARMA(2,2) as a good
model for now.

\begin{Shaded}
\begin{Highlighting}[]
\KeywordTok{tsdisplay}\NormalTok{(sp500_ret, }\DataTypeTok{main=}\StringTok{"S&P500 Returns"}\NormalTok{)}
\end{Highlighting}
\end{Shaded}

\includegraphics{Project_2_files/figure-latex/tsdisplay s\&p500-1.pdf}

The ACF and PACF for S\&P500 monthly returns don't have any significant
spikes at all lags. Our conclusion is that the returns follows white
noise model.

\subsubsection{b. Model fitting}\label{b.-model-fitting}

Since the data that we're dealing with is the monthly returns data, it
is covariance stationary. They have the same mean, or we could say that
they have mean reversion. So both of our data doesn't have any
significant linear trend. The only thing we consider is the seasonality
and cycles component. Now, we will explore the model to find the best
fit.

\paragraph{Model for USD/IDR returns}\label{model-for-usdidr-returns}

\subparagraph{Periodic trend}\label{periodic-trend}

The first model that we propose is the periodic trend for detecting
seasonality. We created the variable time as a predictor and we use
linear regression of returns with respect to sin and cos of time.

\begin{Shaded}
\begin{Highlighting}[]
\CommentTok{# specify the index}
\NormalTok{t =}\StringTok{ }\KeywordTok{seq}\NormalTok{(}\DecValTok{2002} \OperatorTok{+}\StringTok{ }\NormalTok{(}\DecValTok{1}\OperatorTok{/}\DecValTok{12}\NormalTok{), }\DecValTok{2019} \OperatorTok{+}\StringTok{ }\NormalTok{(}\DecValTok{4}\OperatorTok{/}\DecValTok{12}\NormalTok{), }\DataTypeTok{length =} \KeywordTok{length}\NormalTok{(usd_idr_ret))}

\CommentTok{# periodic}
\NormalTok{mod_period_usdidr <-}\StringTok{ }\KeywordTok{lm}\NormalTok{(usd_idr_ret }\OperatorTok{~}\StringTok{ }\KeywordTok{I}\NormalTok{(}\KeywordTok{sin}\NormalTok{(}\DecValTok{2}\OperatorTok{*}\NormalTok{pi}\OperatorTok{*}\NormalTok{t)) }\OperatorTok{+}\StringTok{ }\KeywordTok{I}\NormalTok{(}\KeywordTok{cos}\NormalTok{(}\DecValTok{2}\OperatorTok{*}\NormalTok{pi}\OperatorTok{*}\NormalTok{t)))}
\KeywordTok{summary}\NormalTok{(mod_period_usdidr)}
\end{Highlighting}
\end{Shaded}

\begin{verbatim}
## 
## Call:
## lm(formula = usd_idr_ret ~ I(sin(2 * pi * t)) + I(cos(2 * pi * 
##     t)))
## 
## Residuals:
##      Min       1Q   Median       3Q      Max 
## -14.2753  -1.5022  -0.0644   1.3571  26.9193 
## 
## Coefficients:
##                    Estimate Std. Error t value Pr(>|t|)
## (Intercept)          0.2365     0.2545   0.929    0.354
## I(sin(2 * pi * t))  -0.5318     0.3587  -1.483    0.140
## I(cos(2 * pi * t))  -0.3765     0.3612  -1.042    0.298
## 
## Residual standard error: 3.67 on 205 degrees of freedom
## Multiple R-squared:  0.01583,    Adjusted R-squared:  0.006229 
## F-statistic: 1.649 on 2 and 205 DF,  p-value: 0.1948
\end{verbatim}

\begin{Shaded}
\begin{Highlighting}[]
\KeywordTok{plot}\NormalTok{(usd_idr_ret, }\DataTypeTok{main =} \StringTok{"USD/IDR returns"}\NormalTok{)}
\KeywordTok{lines}\NormalTok{(mod_period_usdidr}\OperatorTok{$}\NormalTok{fit }\OperatorTok{~}\StringTok{ }\NormalTok{t, }\DataTypeTok{col =} \StringTok{"red"}\NormalTok{)}
\end{Highlighting}
\end{Shaded}

\includegraphics{Project_2_files/figure-latex/periodic usd/idr plot-1.pdf}

The result is not statistically significant, independently and jointly.

\subparagraph{Seasonal Dummy trend}\label{seasonal-dummy-trend}

Since we're using the monthly data, we could use dummy variable for each
month to know whether some month is significantly different from the
others.

\begin{Shaded}
\begin{Highlighting}[]
\NormalTok{mod_seasonal_usdidr <-}\StringTok{ }\KeywordTok{tslm}\NormalTok{(usd_idr_ret }\OperatorTok{~}\StringTok{ }\NormalTok{season)}
\KeywordTok{summary}\NormalTok{(mod_seasonal_usdidr)}
\end{Highlighting}
\end{Shaded}

\begin{verbatim}
## 
## Call:
## tslm(formula = usd_idr_ret ~ season)
## 
## Residuals:
##      Min       1Q   Median       3Q      Max 
## -13.9265  -1.3625  -0.0018   1.2764  26.1643 
## 
## Coefficients:
##             Estimate Std. Error t value Pr(>|t|)  
## (Intercept)  -0.6842     0.9025  -0.758   0.4493  
## season2       0.6766     1.2585   0.538   0.5915  
## season3       0.3708     1.2585   0.295   0.7686  
## season4      -0.1924     1.2585  -0.153   0.8786  
## season5       1.0864     1.2585   0.863   0.3891  
## season6       1.0732     1.2764   0.841   0.4015  
## season7       0.6477     1.2764   0.507   0.6124  
## season8       2.1465     1.2764   1.682   0.0942 .
## season9       1.7600     1.2764   1.379   0.1695  
## season10      1.0023     1.2764   0.785   0.4333  
## season11      0.8443     1.2764   0.661   0.5091  
## season12      1.6155     1.2764   1.266   0.2071  
## ---
## Signif. codes:  0 '***' 0.001 '**' 0.01 '*' 0.05 '.' 0.1 ' ' 1
## 
## Residual standard error: 3.721 on 196 degrees of freedom
## Multiple R-squared:  0.03255,    Adjusted R-squared:  -0.02175 
## F-statistic: 0.5994 on 11 and 196 DF,  p-value: 0.828
\end{verbatim}

\begin{Shaded}
\begin{Highlighting}[]
\KeywordTok{plot}\NormalTok{(usd_idr_ret, }\DataTypeTok{main =} \StringTok{"USD/IDR returns"}\NormalTok{)}
\KeywordTok{lines}\NormalTok{(mod_seasonal_usdidr}\OperatorTok{$}\NormalTok{fit, }\DataTypeTok{col =} \StringTok{"red"}\NormalTok{)}
\end{Highlighting}
\end{Shaded}

\includegraphics{Project_2_files/figure-latex/seasonal usd/idr plot-1.pdf}

All seasonal dummy variables are not statistically significant and the
joint hypothesis test is also not statistically significant, therefore
we conclude that there is no seasonal component in USD/IDR returns.

\subparagraph{ARMA (2,2)}\label{arma-22}

Using our first hypothesis, we can try to fit ARMA(2,2) because the ACF
and PACF spikes at lag 2. The function that we're using is arima
function. However, the model object from arima function doesn't tell us
the significant level of the coefficient. So we can create a function to
analyze the significant level before summarizing the model.

\begin{Shaded}
\begin{Highlighting}[]
\CommentTok{# function to give summary for arima model}
\NormalTok{arima_summary <-}\StringTok{ }\ControlFlowTok{function}\NormalTok{(model)\{}
\NormalTok{  coef_arma <-}\StringTok{ }\NormalTok{model}\OperatorTok{$}\NormalTok{coef }\CommentTok{# coefficient}
\NormalTok{  se_arma <-}\StringTok{ }\KeywordTok{sqrt}\NormalTok{(}\KeywordTok{diag}\NormalTok{(model}\OperatorTok{$}\NormalTok{var.coef)) }\CommentTok{# se}
\NormalTok{  t_arma <-}\StringTok{ }\NormalTok{coef_arma }\OperatorTok{/}\StringTok{ }\NormalTok{se_arma }\CommentTok{# t stats}
\NormalTok{  p_arma <-}\StringTok{ }\KeywordTok{pnorm}\NormalTok{(}\KeywordTok{abs}\NormalTok{(t_arma), }\DataTypeTok{lower.tail =} \OtherTok{FALSE}\NormalTok{) }\CommentTok{# p values}
  \KeywordTok{cbind}\NormalTok{(}\StringTok{"coef"}\NormalTok{ =}\StringTok{ }\NormalTok{coef_arma, }\StringTok{"se"}\NormalTok{ =}\StringTok{ }\NormalTok{se_arma, }\StringTok{"t"}\NormalTok{ =}\StringTok{ }\NormalTok{t_arma, }\StringTok{"p"}\NormalTok{ =}\StringTok{ }\KeywordTok{round}\NormalTok{(p_arma, }\DecValTok{5}\NormalTok{))}
\NormalTok{\}}

\NormalTok{mod_arma_22_usdidr <-}\StringTok{ }\KeywordTok{arima}\NormalTok{(usd_idr_ret, }\DataTypeTok{order =} \KeywordTok{c}\NormalTok{(}\DecValTok{2}\NormalTok{,}\DecValTok{0}\NormalTok{,}\DecValTok{2}\NormalTok{)) }
\KeywordTok{arima_summary}\NormalTok{(mod_arma_22_usdidr)}
\end{Highlighting}
\end{Shaded}

\begin{verbatim}
##                 coef        se         t       p
## ar1        0.5942252 0.1288501  4.611756 0.00000
## ar2       -0.7806750 0.1129330 -6.912725 0.00000
## ma1       -0.7478787 0.1376726 -5.432300 0.00000
## ma2        0.7849849 0.1085850  7.229218 0.00000
## intercept  0.2233734 0.2153731  1.037147 0.14983
\end{verbatim}

It turns out that ARMA (2,2) does a great job. We can see that all the
coefficients are statistically significant from the p-values.

\subparagraph{ARMA (p,q)}\label{arma-pq}

Another thing to consider is fitting another combination (p,q) of ARMA
model. To do that, we use for loop to fit all combinations of the order
of ARMA, then we compare the model using AIC as our measurement. We
first creates an empty matrix to store the AIC for each combination.

\begin{Shaded}
\begin{Highlighting}[]
\CommentTok{#For loop to find best ARMA(p,q)}
\NormalTok{m =}\StringTok{ }\DecValTok{4} \CommentTok{# combination for p = 1,2,3,4 with q = 1,2,3,4}

\NormalTok{AIC_mat <-}\StringTok{ }\KeywordTok{matrix}\NormalTok{(}\OtherTok{NA}\NormalTok{, }\DataTypeTok{ncol =}\NormalTok{ m, }\DataTypeTok{nrow =}\NormalTok{ m)}
\KeywordTok{rownames}\NormalTok{(AIC_mat) <-}\StringTok{ }\DecValTok{1}\OperatorTok{:}\NormalTok{m}
\KeywordTok{colnames}\NormalTok{(AIC_mat) <-}\StringTok{ }\DecValTok{1}\OperatorTok{:}\NormalTok{m}

\ControlFlowTok{for}\NormalTok{(i }\ControlFlowTok{in} \DecValTok{1}\OperatorTok{:}\NormalTok{m)\{}
  \ControlFlowTok{for}\NormalTok{(j }\ControlFlowTok{in} \DecValTok{1}\OperatorTok{:}\NormalTok{m)\{}
\NormalTok{    AIC_mat[i,j] <-}\StringTok{ }\KeywordTok{arima}\NormalTok{(usd_idr_ret, }\DataTypeTok{order =} \KeywordTok{c}\NormalTok{(i,}\DecValTok{0}\NormalTok{,j))}\OperatorTok{$}\NormalTok{aic}
\NormalTok{  \}}
\NormalTok{\}}


\NormalTok{AIC_mat }\CommentTok{# All AIC's }
\end{Highlighting}
\end{Shaded}

\begin{verbatim}
##          1        2        3        4
## 1 1135.199 1134.382 1134.395 1132.235
## 2 1132.849 1129.949 1131.145 1134.142
## 3 1133.961 1131.264 1132.851 1134.619
## 4 1133.953 1132.894 1134.625 1134.432
\end{verbatim}

We can see that ARMA(2,2) has the smallest AIC compared to the others,
we could also try ARMA(2,3) and ARMA(3,2) because their AIC are closed
compared to ARMA(2,2)

ARMA(2,3)

\begin{Shaded}
\begin{Highlighting}[]
\NormalTok{mod_arma_23_usdidr <-}\StringTok{ }\KeywordTok{arima}\NormalTok{(usd_idr_ret, }\DataTypeTok{order =} \KeywordTok{c}\NormalTok{(}\DecValTok{2}\NormalTok{,}\DecValTok{0}\NormalTok{,}\DecValTok{3}\NormalTok{)) }
\KeywordTok{arima_summary}\NormalTok{(mod_arma_23_usdidr) }
\end{Highlighting}
\end{Shaded}

\begin{verbatim}
##                  coef         se          t       p
## ar1        0.56566955 0.16309572  3.4683287 0.00026
## ar2       -0.66843872 0.18598669 -3.5940137 0.00016
## ma1       -0.68332087 0.17443015 -3.9174470 0.00004
## ma2        0.62318022 0.22549695  2.7635860 0.00286
## ma3        0.09139671 0.09619065  0.9501621 0.17101
## intercept  0.22074499 0.22982845  0.9604772 0.16841
\end{verbatim}

Here the coefficient for ma3 is not significant.

ARMA(3,2)

\begin{Shaded}
\begin{Highlighting}[]
\NormalTok{mod_arma_32_usdidr <-}\StringTok{ }\KeywordTok{arima}\NormalTok{(usd_idr_ret, }\DataTypeTok{order =} \KeywordTok{c}\NormalTok{(}\DecValTok{3}\NormalTok{,}\DecValTok{0}\NormalTok{,}\DecValTok{2}\NormalTok{)) }
\KeywordTok{arima_summary}\NormalTok{(mod_arma_32_usdidr) }
\end{Highlighting}
\end{Shaded}

\begin{verbatim}
##                  coef        se          t       p
## ar1        0.67277727 0.1661727  4.0486631 0.00003
## ar2       -0.74630895 0.1374486 -5.4297303 0.00000
## ar3        0.07826766 0.0913042  0.8572186 0.19566
## ma1       -0.79515349 0.1534973 -5.1802454 0.00000
## ma2        0.72405793 0.1526050  4.7446528 0.00000
## intercept  0.22104547 0.2294538  0.9633549 0.16768
\end{verbatim}

And here, the coefficient for ar3 is not statistically significant.

For now, we conclude that ARMA(2,2) is the best model because all
variables are statistically significant.

\subparagraph{Auto Arima}\label{auto-arima}

To check what R think the best model is, we can use auto arima function.

\begin{Shaded}
\begin{Highlighting}[]
\NormalTok{mod_autoarima_usdidr =}\StringTok{ }\KeywordTok{auto.arima}\NormalTok{(usd_idr_ret)}
\NormalTok{mod_autoarima_usdidr}
\end{Highlighting}
\end{Shaded}

\begin{verbatim}
## Series: usd_idr_ret 
## ARIMA(3,0,2) with zero mean 
## 
## Coefficients:
##          ar1      ar2     ar3      ma1     ma2
##       0.6864  -0.7438  0.0866  -0.8039  0.7213
## s.e.  0.1645   0.1400  0.0906   0.1521  0.1539
## 
## sigma^2 estimated as 12.95:  log likelihood=-559.09
## AIC=1130.18   AICc=1130.6   BIC=1150.2
\end{verbatim}

The auto arima agrees with ARMA(2,2) with additional seasonal AR(1).
Next, we check the significant level for the seasonal ar coefficient.

\begin{Shaded}
\begin{Highlighting}[]
\KeywordTok{arima_summary}\NormalTok{(mod_autoarima_usdidr)}
\end{Highlighting}
\end{Shaded}

\begin{verbatim}
##            coef         se          t       p
## ar1  0.68637374 0.16446880  4.1732763 0.00002
## ar2 -0.74375152 0.13997168 -5.3135857 0.00000
## ar3  0.08660883 0.09063615  0.9555661 0.16965
## ma1 -0.80386159 0.15209941 -5.2851065 0.00000
## ma2  0.72127158 0.15389652  4.6867309 0.00000
\end{verbatim}

The s-ar(1) coefficient is not statistically significant. We could also
compare the AIC with ARMA(2,2) to check which one is better.

\begin{Shaded}
\begin{Highlighting}[]
\KeywordTok{AIC}\NormalTok{(mod_arma_22_usdidr, mod_autoarima_usdidr)}
\end{Highlighting}
\end{Shaded}

\begin{verbatim}
##                      df      AIC
## mod_arma_22_usdidr    6 1129.949
## mod_autoarima_usdidr  6 1130.177
\end{verbatim}

The AIC shows that ARMA(2,2) is still a better model than ARMA(2,2) +
S-AR(1). Therefore we will use ARMA(2,2) model for USD/IDR returns.

\paragraph{Model for S\&P500 returns}\label{model-for-sp500-returns}

Since the ACF and PACF of this data tells us that it is white noise, we
can't make a guess. However, since the two dataset are somewhat have
similar characteristic, we would choose ARMA(2,2) for the S\&P500 as
well.

\subparagraph{ARMA(2,2)}\label{arma22}

We analyze the significance of ARMA(2,2) coefficients in the similar way
as the USD/IDR,

\begin{Shaded}
\begin{Highlighting}[]
\NormalTok{mod_arma_22_sp500 <-}\StringTok{ }\KeywordTok{arima}\NormalTok{(sp500_ret, }\DataTypeTok{order =} \KeywordTok{c}\NormalTok{(}\DecValTok{2}\NormalTok{,}\DecValTok{0}\NormalTok{,}\DecValTok{2}\NormalTok{)) }
\KeywordTok{summary}\NormalTok{(mod_arma_22_sp500)}
\end{Highlighting}
\end{Shaded}

\begin{verbatim}
## 
## Call:
## arima(x = sp500_ret, order = c(2, 0, 2))
## 
## Coefficients:
##           ar1      ar2     ma1     ma2  intercept
##       -1.4880  -0.8523  1.5186  0.8498     0.5203
## s.e.   0.3931   0.5264  0.3575  0.5246     0.2819
## 
## sigma^2 estimated as 16.25:  log likelihood = -585.17,  aic = 1182.33
## 
## Training set error measures:
##                       ME     RMSE      MAE      MPE    MAPE      MASE
## Training set 0.003601198 4.031707 2.992703 58.15256 190.398 0.7301812
##                    ACF1
## Training set 0.07089607
\end{verbatim}

The standard errors are NaN in this case so we cannot analyze the
significance of each coefficient parameters.

\subparagraph{Auto Arima}\label{auto-arima-1}

Then, we cross-check with auto arima to see which one is better,

\begin{Shaded}
\begin{Highlighting}[]
\NormalTok{mod_autoarima_sp500 =}\StringTok{ }\KeywordTok{auto.arima}\NormalTok{(sp500_ret)}
\NormalTok{mod_autoarima_sp500}
\end{Highlighting}
\end{Shaded}

\begin{verbatim}
## Series: sp500_ret 
## ARIMA(2,0,2)(1,0,0)[12] with non-zero mean 
## 
## Coefficients:
##           ar1      ar2     ma1     ma2     sar1    mean
##       -0.2600  -0.9333  0.3413  0.9380  -0.0265  0.5251
## s.e.   0.0507   0.0453  0.0428  0.0532   0.0800  0.2776
## 
## sigma^2 estimated as 16.03:  log likelihood=-580.97
## AIC=1175.93   AICc=1176.49   BIC=1199.3
\end{verbatim}

\begin{Shaded}
\begin{Highlighting}[]
\KeywordTok{arima_summary}\NormalTok{(mod_autoarima_sp500)}
\end{Highlighting}
\end{Shaded}

\begin{verbatim}
##                  coef         se           t       p
## ar1       -0.25999329 0.05065603  -5.1325238 0.00000
## ar2       -0.93334704 0.04530785 -20.6001185 0.00000
## ma1        0.34134679 0.04284832   7.9663986 0.00000
## ma2        0.93795430 0.05316619  17.6419310 0.00000
## sar1      -0.02652206 0.07995573  -0.3317093 0.37005
## intercept  0.52508869 0.27761985   1.8913946 0.02929
\end{verbatim}

We can see that the coefficients for AR and MA order 1,2 are
statistically significant. However, S-AR(1) is not statistically
significant. ARMA(2,2) may be a good model.

\paragraph{Analyzing GARCH(p,q)}\label{analyzing-garchpq}

Although our current model, ARMA(2,2), may be good, we still dont know
if GARCH is needed to better fit the model.

\subparagraph{GARCH(p,q) for USD/IDR}\label{garchpq-for-usdidr}

First, we look at the residuals of the USD/IDR ARMA(2,2),

\begin{Shaded}
\begin{Highlighting}[]
\KeywordTok{tsdisplay}\NormalTok{(mod_arma_22_usdidr}\OperatorTok{$}\NormalTok{res)}
\end{Highlighting}
\end{Shaded}

\includegraphics{Project_2_files/figure-latex/unnamed-chunk-3-1.pdf}

It seems that the residuals are all almost zero which is a good signal.
Now, we square the residuals to detect any need for GARCH.

\begin{Shaded}
\begin{Highlighting}[]
\CommentTok{#Analyze GARCH}
\KeywordTok{tsdisplay}\NormalTok{(mod_arma_22_usdidr}\OperatorTok{$}\NormalTok{res}\OperatorTok{*}\NormalTok{mod_arma_22_usdidr}\OperatorTok{$}\NormalTok{res)}
\end{Highlighting}
\end{Shaded}

\includegraphics{Project_2_files/figure-latex/unnamed-chunk-4-1.pdf}

We can see that in the squared residuals, there are still spikes which
indicates the need for GARCH model. We now find the best p and q for
GARCH(p,q) using for loop and compare their MSE and MAE.

\begin{Shaded}
\begin{Highlighting}[]
\CommentTok{#Use for loop to find best GARCH(p,q)}
\NormalTok{m <-}\StringTok{ }\DecValTok{6}
\NormalTok{garch_mse_mat <-}\StringTok{ }\KeywordTok{matrix}\NormalTok{(}\OtherTok{NA}\NormalTok{, }\DataTypeTok{ncol =}\NormalTok{ m, }\DataTypeTok{nrow =}\NormalTok{ m)}
\NormalTok{garch_mae_mat <-}\StringTok{ }\KeywordTok{matrix}\NormalTok{(}\OtherTok{NA}\NormalTok{, }\DataTypeTok{ncol =}\NormalTok{ m, }\DataTypeTok{nrow =}\NormalTok{ m)}
\ControlFlowTok{for}\NormalTok{ (i }\ControlFlowTok{in} \DecValTok{1}\OperatorTok{:}\NormalTok{m)\{}
  \ControlFlowTok{for}\NormalTok{(j }\ControlFlowTok{in} \DecValTok{1}\OperatorTok{:}\NormalTok{m)\{}
\NormalTok{    model <-}\StringTok{ }\KeywordTok{ugarchspec}\NormalTok{(}
      \DataTypeTok{variance.model =} \KeywordTok{list}\NormalTok{(}\DataTypeTok{model =} \StringTok{"sGARCH"}\NormalTok{, }\DataTypeTok{garchOrder =} \KeywordTok{c}\NormalTok{(i, j)),}
      \DataTypeTok{mean.model =} \KeywordTok{list}\NormalTok{(}\DataTypeTok{armaOrder =} \KeywordTok{c}\NormalTok{(}\DecValTok{2}\NormalTok{, }\DecValTok{2}\NormalTok{), }\DataTypeTok{include.mean =} \OtherTok{TRUE}\NormalTok{),}
      \DataTypeTok{distribution.model =} \StringTok{"sstd"}\NormalTok{)}
    
\NormalTok{    modelfit <-}\StringTok{ }\KeywordTok{ugarchfit}\NormalTok{(}\DataTypeTok{spec=}\NormalTok{model,}\DataTypeTok{data=}\NormalTok{usd_idr_ret)}
    
\NormalTok{    a <-}\StringTok{ }\KeywordTok{attributes}\NormalTok{(modelfit)}
    
\NormalTok{    garch_mse_mat[i,j] <-}\StringTok{ }\KeywordTok{mse}\NormalTok{(usd_idr_ret, a}\OperatorTok{$}\NormalTok{fit}\OperatorTok{$}\NormalTok{fitted.values)}
\NormalTok{    garch_mae_mat[i,j] <-}\StringTok{ }\KeywordTok{mae}\NormalTok{(usd_idr_ret, a}\OperatorTok{$}\NormalTok{fit}\OperatorTok{$}\NormalTok{fitted.values)}
    
\NormalTok{  \}}
\NormalTok{\}}

\NormalTok{garch_mae_mat}
\end{Highlighting}
\end{Shaded}

\begin{verbatim}
##          [,1]     [,2]     [,3]     [,4]     [,5]     [,6]
## [1,] 2.133675 2.134206 2.134818 2.135488 2.133607 2.133689
## [2,] 2.133674 2.134208 2.135003 2.143408 2.135454 2.134794
## [3,] 2.133489 2.133868 2.135004 2.143408 2.135879 2.134791
## [4,] 2.133061 2.132980 2.134151 2.143409 2.135884 2.134790
## [5,] 2.136912 2.136917 2.133900 2.144341 2.135941 2.133958
## [6,] 2.143552 2.143552 2.143548 2.143556 2.143545 2.143558
\end{verbatim}

\begin{Shaded}
\begin{Highlighting}[]
\NormalTok{garch_mse_mat}
\end{Highlighting}
\end{Shaded}

\begin{verbatim}
##          [,1]     [,2]     [,3]     [,4]     [,5]     [,6]
## [1,] 12.79735 12.81023 12.80698 12.82717 12.82299 12.84104
## [2,] 12.79739 12.81022 12.79736 12.83389 12.84214 12.84035
## [3,] 12.79763 12.81009 12.79738 12.83388 12.84734 12.84033
## [4,] 12.81302 12.82044 12.80967 12.83390 12.84736 12.84032
## [5,] 12.81934 12.81937 12.80673 12.83951 12.84530 12.83775
## [6,] 12.87486 12.87486 12.87486 12.87490 12.87483 12.87491
\end{verbatim}

\begin{Shaded}
\begin{Highlighting}[]
\KeywordTok{which}\NormalTok{(garch_mae_mat }\OperatorTok{==}\StringTok{ }\KeywordTok{min}\NormalTok{(garch_mae_mat), }\DataTypeTok{arr.ind =} \OtherTok{TRUE}\NormalTok{)}
\end{Highlighting}
\end{Shaded}

\begin{verbatim}
##      row col
## [1,]   4   2
\end{verbatim}

\begin{Shaded}
\begin{Highlighting}[]
\KeywordTok{which}\NormalTok{(garch_mse_mat }\OperatorTok{==}\StringTok{ }\KeywordTok{min}\NormalTok{(garch_mse_mat), }\DataTypeTok{arr.ind =} \OtherTok{TRUE}\NormalTok{)}
\end{Highlighting}
\end{Shaded}

\begin{verbatim}
##      row col
## [1,]   1   1
\end{verbatim}

From the for loop summary, we find that the best value for p,q is 4 and
1 respectively. Thus, we use GARCH(4,1) for the model.

\begin{Shaded}
\begin{Highlighting}[]
\NormalTok{garch_mod=}\KeywordTok{ugarchspec}\NormalTok{(}
  \DataTypeTok{variance.model =} \KeywordTok{list}\NormalTok{(}\DataTypeTok{model =} \StringTok{"sGARCH"}\NormalTok{, }\DataTypeTok{garchOrder =} \KeywordTok{c}\NormalTok{(}\DecValTok{4}\NormalTok{, }\DecValTok{1}\NormalTok{)),}
  \DataTypeTok{mean.model =} \KeywordTok{list}\NormalTok{(}\DataTypeTok{armaOrder =} \KeywordTok{c}\NormalTok{(}\DecValTok{2}\NormalTok{, }\DecValTok{2}\NormalTok{), }\DataTypeTok{include.mean =} \OtherTok{TRUE}\NormalTok{),}
  \DataTypeTok{distribution.model =} \StringTok{"sstd"}\NormalTok{)}

\CommentTok{# Fit the model}
\NormalTok{garchfit41_usdidr =}\StringTok{ }\KeywordTok{ugarchfit}\NormalTok{(}\DataTypeTok{spec=}\NormalTok{garch_mod, }\DataTypeTok{data=}\NormalTok{usd_idr_ret)}
\NormalTok{garchfit41_usdidr}
\end{Highlighting}
\end{Shaded}

\begin{verbatim}
## 
## *---------------------------------*
## *          GARCH Model Fit        *
## *---------------------------------*
## 
## Conditional Variance Dynamics    
## -----------------------------------
## GARCH Model  : sGARCH(4,1)
## Mean Model   : ARFIMA(2,0,2)
## Distribution : sstd 
## 
## Optimal Parameters
## ------------------------------------
##         Estimate  Std. Error   t value Pr(>|t|)
## mu      0.266143    0.131951  2.016984 0.043697
## ar1     0.527715    0.100986  5.225619 0.000000
## ar2    -0.709405    0.132573 -5.351073 0.000000
## ma1    -0.671211    0.115201 -5.826415 0.000000
## ma2     0.751433    0.103173  7.283198 0.000000
## omega   1.270240    1.374173  0.924367 0.355295
## alpha1  0.644136    0.231406  2.783574 0.005376
## alpha2  0.000000    0.499419  0.000000 1.000000
## alpha3  0.000000    0.237680  0.000000 1.000000
## alpha4  0.006715    0.105199  0.063827 0.949108
## beta1   0.348149    0.409986  0.849172 0.395786
## skew    1.121446    0.094075 11.920795 0.000000
## shape   4.342739    1.193692  3.638074 0.000275
## 
## Robust Standard Errors:
##         Estimate  Std. Error   t value Pr(>|t|)
## mu      0.266143    0.180912  1.471122 0.141258
## ar1     0.527715    0.111614  4.728051 0.000002
## ar2    -0.709405    0.200267 -3.542300 0.000397
## ma1    -0.671211    0.161982 -4.143737 0.000034
## ma2     0.751433    0.147575  5.091865 0.000000
## omega   1.270240    4.836345  0.262645 0.792824
## alpha1  0.644136    0.309526  2.081043 0.037430
## alpha2  0.000000    1.608281  0.000000 1.000000
## alpha3  0.000000    0.605965  0.000000 1.000000
## alpha4  0.006715    0.231442  0.029012 0.976855
## beta1   0.348149    1.917842  0.181531 0.855950
## skew    1.121446    0.093506 11.993297 0.000000
## shape   4.342739    1.473411  2.947405 0.003205
## 
## LogLikelihood : -479.8734 
## 
## Information Criteria
## ------------------------------------
##                    
## Akaike       4.7392
## Bayes        4.9478
## Shibata      4.7320
## Hannan-Quinn 4.8235
## 
## Weighted Ljung-Box Test on Standardized Residuals
## ------------------------------------
##                          statistic p-value
## Lag[1]                       4.383 0.03631
## Lag[2*(p+q)+(p+q)-1][11]     6.491 0.20443
## Lag[4*(p+q)+(p+q)-1][19]     9.282 0.59049
## d.o.f=4
## H0 : No serial correlation
## 
## Weighted Ljung-Box Test on Standardized Squared Residuals
## ------------------------------------
##                          statistic p-value
## Lag[1]                      0.6412  0.4233
## Lag[2*(p+q)+(p+q)-1][14]    2.0079  0.9910
## Lag[4*(p+q)+(p+q)-1][24]    3.4948  0.9989
## d.o.f=5
## 
## Weighted ARCH LM Tests
## ------------------------------------
##              Statistic Shape Scale P-Value
## ARCH Lag[6]    0.04648 0.500 2.000  0.8293
## ARCH Lag[8]    0.30961 1.480 1.774  0.9480
## ARCH Lag[10]   0.59016 2.424 1.650  0.9790
## 
## Nyblom stability test
## ------------------------------------
## Joint Statistic:  2.9865
## Individual Statistics:              
## mu     0.49503
## ar1    0.19737
## ar2    0.13984
## ma1    0.28582
## ma2    0.07738
## omega  0.34806
## alpha1 0.05219
## alpha2 0.04413
## alpha3 0.04814
## alpha4 0.05778
## beta1  0.08343
## skew   0.26815
## shape  0.06576
## 
## Asymptotic Critical Values (10% 5% 1%)
## Joint Statistic:          2.89 3.15 3.69
## Individual Statistic:     0.35 0.47 0.75
## 
## Sign Bias Test
## ------------------------------------
##                    t-value   prob sig
## Sign Bias           1.5102 0.1325    
## Negative Sign Bias  0.5809 0.5620    
## Positive Sign Bias  0.2341 0.8151    
## Joint Effect        2.4818 0.4786    
## 
## 
## Adjusted Pearson Goodness-of-Fit Test:
## ------------------------------------
##   group statistic p-value(g-1)
## 1    20     12.77       0.8502
## 2    30     19.02       0.9208
## 3    40     29.69       0.8587
## 4    50     41.04       0.7835
## 
## 
## Elapsed time : 0.3069692
\end{verbatim}

\begin{Shaded}
\begin{Highlighting}[]
\NormalTok{att_garch_usdidr =}\StringTok{ }\KeywordTok{attributes}\NormalTok{(garchfit41_usdidr)}
 \CommentTok{# standardized residuals}
\KeywordTok{tsdisplay}\NormalTok{((att_garch_usdidr}\OperatorTok{$}\NormalTok{fit}\OperatorTok{$}\NormalTok{residuals}\OperatorTok{*}\NormalTok{att_garch_usdidr}\OperatorTok{$}\NormalTok{fit}\OperatorTok{$}\NormalTok{residuals)}\OperatorTok{/}\NormalTok{att_garch_usdidr}\OperatorTok{$}\NormalTok{fit}\OperatorTok{$}\NormalTok{var,}
          \DataTypeTok{main =} \StringTok{"Standardized Residuals for GARCH(4,1)"}\NormalTok{)}
\end{Highlighting}
\end{Shaded}

\includegraphics{Project_2_files/figure-latex/unnamed-chunk-6-1.pdf}

From the ACF PACF we can clearly see that the GARCH model eliminates the
remaining spikes of the residual. Thus, we use GARCH(4,1)

\subparagraph{GARCH(p,q) for S\&P500}\label{garchpq-for-sp500}

Now, we will find the best GARCH(p,q) value for S\&P500 using the same
approach. First, we look at the residual and squared residuals.

\begin{Shaded}
\begin{Highlighting}[]
\KeywordTok{tsdisplay}\NormalTok{(mod_arma_22_usdidr}\OperatorTok{$}\NormalTok{res)}
\end{Highlighting}
\end{Shaded}

\includegraphics{Project_2_files/figure-latex/unnamed-chunk-7-1.pdf}

\begin{Shaded}
\begin{Highlighting}[]
\KeywordTok{tsdisplay}\NormalTok{(mod_arma_22_usdidr}\OperatorTok{$}\NormalTok{res}\OperatorTok{*}\NormalTok{mod_arma_22_usdidr}\OperatorTok{$}\NormalTok{res)}
\end{Highlighting}
\end{Shaded}

\includegraphics{Project_2_files/figure-latex/unnamed-chunk-7-2.pdf}

We see similar pattern as before, there are still spikes in the model.
Thus, we look for GARCH.

\begin{Shaded}
\begin{Highlighting}[]
\NormalTok{m <-}\StringTok{ }\DecValTok{6}
\NormalTok{garch_mse_mat <-}\StringTok{ }\KeywordTok{matrix}\NormalTok{(}\OtherTok{NA}\NormalTok{, }\DataTypeTok{ncol =}\NormalTok{ m, }\DataTypeTok{nrow =}\NormalTok{ m)}
\NormalTok{garch_mae_mat <-}\StringTok{ }\KeywordTok{matrix}\NormalTok{(}\OtherTok{NA}\NormalTok{, }\DataTypeTok{ncol =}\NormalTok{ m, }\DataTypeTok{nrow =}\NormalTok{ m)}
\ControlFlowTok{for}\NormalTok{ (i }\ControlFlowTok{in} \DecValTok{1}\OperatorTok{:}\NormalTok{m)\{}
  \ControlFlowTok{for}\NormalTok{(j }\ControlFlowTok{in} \DecValTok{1}\OperatorTok{:}\NormalTok{m)\{}
\NormalTok{    model <-}\StringTok{ }\KeywordTok{ugarchspec}\NormalTok{(}
      \DataTypeTok{variance.model =} \KeywordTok{list}\NormalTok{(}\DataTypeTok{model =} \StringTok{"sGARCH"}\NormalTok{, }\DataTypeTok{garchOrder =} \KeywordTok{c}\NormalTok{(i, j)),}
      \DataTypeTok{mean.model =} \KeywordTok{list}\NormalTok{(}\DataTypeTok{armaOrder =} \KeywordTok{c}\NormalTok{(}\DecValTok{2}\NormalTok{, }\DecValTok{2}\NormalTok{), }\DataTypeTok{include.mean =} \OtherTok{TRUE}\NormalTok{),}
      \DataTypeTok{distribution.model =} \StringTok{"sstd"}\NormalTok{)}
    
\NormalTok{    modelfit <-}\StringTok{ }\KeywordTok{ugarchfit}\NormalTok{(}\DataTypeTok{spec=}\NormalTok{model,}\DataTypeTok{data=}\NormalTok{sp500_ret)}
    
\NormalTok{    a <-}\StringTok{ }\KeywordTok{attributes}\NormalTok{(modelfit)}
    
\NormalTok{    garch_mse_mat[i,j] <-}\StringTok{ }\KeywordTok{mse}\NormalTok{(sp500_ret, a}\OperatorTok{$}\NormalTok{fit}\OperatorTok{$}\NormalTok{fitted.values)}
\NormalTok{    garch_mae_mat[i,j] <-}\StringTok{ }\KeywordTok{mae}\NormalTok{(sp500_ret, a}\OperatorTok{$}\NormalTok{fit}\OperatorTok{$}\NormalTok{fitted.values)}
    
\NormalTok{  \}}
\NormalTok{\}}

\NormalTok{garch_mae_mat}
\end{Highlighting}
\end{Shaded}

\begin{verbatim}
##          [,1]     [,2]     [,3]     [,4]     [,5]     [,6]
## [1,] 2.991512 2.943396 2.942555 2.943204 2.942970 2.943431
## [2,] 3.013499 2.937167 2.986919 2.937071 2.946621 2.994040
## [3,] 2.987362 2.987067 2.938959 2.989354 2.992827 2.939377
## [4,] 2.943035 2.942203 2.987123 2.989339 2.992834 2.995020
## [5,] 2.988924 2.942382 2.988132 2.942414 2.992793 2.995028
## [6,] 2.946121 2.949412 2.997425 2.997409 2.993069 2.997545
\end{verbatim}

\begin{Shaded}
\begin{Highlighting}[]
\NormalTok{garch_mse_mat}
\end{Highlighting}
\end{Shaded}

\begin{verbatim}
##          [,1]     [,2]     [,3]     [,4]     [,5]     [,6]
## [1,] 16.52165 16.08806 16.09222 16.09005 16.09269 16.08826
## [2,] 17.12588 16.07988 16.52122 16.07959 16.11396 16.55184
## [3,] 16.50239 16.48584 16.07816 16.50455 16.52408 16.07845
## [4,] 16.07193 16.07088 16.52709 16.50485 16.52438 16.55957
## [5,] 16.51204 16.06874 16.52534 16.06832 16.52459 16.55952
## [6,] 16.06714 16.08208 16.48486 16.48486 16.51965 16.55959
\end{verbatim}

\begin{Shaded}
\begin{Highlighting}[]
\KeywordTok{which}\NormalTok{(garch_mae_mat }\OperatorTok{==}\StringTok{ }\KeywordTok{min}\NormalTok{(garch_mae_mat), }\DataTypeTok{arr.ind =} \OtherTok{TRUE}\NormalTok{)}
\end{Highlighting}
\end{Shaded}

\begin{verbatim}
##      row col
## [1,]   2   4
\end{verbatim}

\begin{Shaded}
\begin{Highlighting}[]
\KeywordTok{which}\NormalTok{(garch_mse_mat }\OperatorTok{==}\StringTok{ }\KeywordTok{min}\NormalTok{(garch_mse_mat), }\DataTypeTok{arr.ind =} \OtherTok{TRUE}\NormalTok{)}
\end{Highlighting}
\end{Shaded}

\begin{verbatim}
##      row col
## [1,]   6   1
\end{verbatim}

We can see that the smallest error in the analysis above is at
GARCH(2,1) and GARCH(2,6). Thus, in this case we use GARCH(2,1) as it is
a simpler model with less parameters.

\begin{Shaded}
\begin{Highlighting}[]
\NormalTok{garch_mod=}\KeywordTok{ugarchspec}\NormalTok{(}
  \DataTypeTok{variance.model =} \KeywordTok{list}\NormalTok{(}\DataTypeTok{model =} \StringTok{"sGARCH"}\NormalTok{, }\DataTypeTok{garchOrder =} \KeywordTok{c}\NormalTok{(}\DecValTok{2}\NormalTok{, }\DecValTok{1}\NormalTok{)),}
  \DataTypeTok{mean.model =} \KeywordTok{list}\NormalTok{(}\DataTypeTok{armaOrder =} \KeywordTok{c}\NormalTok{(}\DecValTok{2}\NormalTok{, }\DecValTok{2}\NormalTok{), }\DataTypeTok{include.mean =} \OtherTok{TRUE}\NormalTok{),}
  \DataTypeTok{distribution.model =} \StringTok{"sstd"}\NormalTok{)}

\CommentTok{# Fit the model}
\NormalTok{garchfit21_sp500 =}\StringTok{ }\KeywordTok{ugarchfit}\NormalTok{(}\DataTypeTok{spec=}\NormalTok{garch_mod, }\DataTypeTok{data=}\NormalTok{sp500_ret)}
\NormalTok{garchfit21_sp500}
\end{Highlighting}
\end{Shaded}

\begin{verbatim}
## 
## *---------------------------------*
## *          GARCH Model Fit        *
## *---------------------------------*
## 
## Conditional Variance Dynamics    
## -----------------------------------
## GARCH Model  : sGARCH(2,1)
## Mean Model   : ARFIMA(2,0,2)
## Distribution : sstd 
## 
## Optimal Parameters
## ------------------------------------
##         Estimate  Std. Error  t value Pr(>|t|)
## mu       0.81037    0.187949  4.31165 0.000016
## ar1     -0.91829    0.203255 -4.51793 0.000006
## ar2     -0.63845    0.145851 -4.37737 0.000012
## ma1      0.80230    0.206904  3.87763 0.000105
## ma2      0.59329    0.167591  3.54010 0.000400
## omega    1.81344    0.906046  2.00149 0.045340
## alpha1   0.16392    0.074366  2.20416 0.027513
## alpha2   0.18618    0.139206  1.33745 0.181075
## beta1    0.52794    0.153799  3.43267 0.000598
## skew     0.64422    0.097111  6.63387 0.000000
## shape   59.99871  139.515459  0.43005 0.667159
## 
## Robust Standard Errors:
##         Estimate  Std. Error  t value Pr(>|t|)
## mu       0.81037    0.174789  4.63629 0.000004
## ar1     -0.91829    0.100470 -9.13998 0.000000
## ar2     -0.63845    0.092131 -6.92976 0.000000
## ma1      0.80230    0.117328  6.83808 0.000000
## ma2      0.59329    0.123355  4.80959 0.000002
## omega    1.81344    1.061741  1.70799 0.087639
## alpha1   0.16392    0.068356  2.39797 0.016486
## alpha2   0.18618    0.152434  1.22139 0.221940
## beta1    0.52794    0.174976  3.01723 0.002551
## skew     0.64422    0.102232  6.30160 0.000000
## shape   59.99871   72.400028  0.82871 0.407268
## 
## LogLikelihood : -551.6404 
## 
## Information Criteria
## ------------------------------------
##                    
## Akaike       5.4100
## Bayes        5.5865
## Shibata      5.4048
## Hannan-Quinn 5.4814
## 
## Weighted Ljung-Box Test on Standardized Residuals
## ------------------------------------
##                          statistic p-value
## Lag[1]                       2.982  0.0842
## Lag[2*(p+q)+(p+q)-1][11]     5.741  0.6564
## Lag[4*(p+q)+(p+q)-1][19]     9.153  0.6124
## d.o.f=4
## H0 : No serial correlation
## 
## Weighted Ljung-Box Test on Standardized Squared Residuals
## ------------------------------------
##                          statistic p-value
## Lag[1]                      0.3271  0.5674
## Lag[2*(p+q)+(p+q)-1][8]     2.4989  0.7761
## Lag[4*(p+q)+(p+q)-1][14]    5.2857  0.7334
## d.o.f=3
## 
## Weighted ARCH LM Tests
## ------------------------------------
##             Statistic Shape Scale P-Value
## ARCH Lag[4]     2.225 0.500 2.000  0.1358
## ARCH Lag[6]     2.727 1.461 1.711  0.3506
## ARCH Lag[8]     3.097 2.368 1.583  0.5254
## 
## Nyblom stability test
## ------------------------------------
## Joint Statistic:  2.2809
## Individual Statistics:              
## mu     0.07771
## ar1    0.21997
## ar2    0.08912
## ma1    0.23972
## ma2    0.18989
## omega  0.05436
## alpha1 0.04584
## alpha2 0.03764
## beta1  0.06020
## skew   0.06990
## shape  0.25703
## 
## Asymptotic Critical Values (10% 5% 1%)
## Joint Statistic:          2.49 2.75 3.27
## Individual Statistic:     0.35 0.47 0.75
## 
## Sign Bias Test
## ------------------------------------
##                    t-value    prob sig
## Sign Bias           0.6089 0.54328    
## Negative Sign Bias  1.1907 0.23517    
## Positive Sign Bias  2.1757 0.03073  **
## Joint Effect        9.2070 0.02666  **
## 
## 
## Adjusted Pearson Goodness-of-Fit Test:
## ------------------------------------
##   group statistic p-value(g-1)
## 1    20     15.46       0.6928
## 2    30     29.69       0.4295
## 3    40     37.00       0.5614
## 4    50     51.13       0.3898
## 
## 
## Elapsed time : 0.3734601
\end{verbatim}

\begin{Shaded}
\begin{Highlighting}[]
\NormalTok{att_garch_sp500 =}\StringTok{ }\KeywordTok{attributes}\NormalTok{(garchfit21_sp500)}
\CommentTok{# standardized residuals}
\KeywordTok{tsdisplay}\NormalTok{((att_garch_sp500}\OperatorTok{$}\NormalTok{fit}\OperatorTok{$}\NormalTok{residuals}\OperatorTok{*}\NormalTok{att_garch_sp500}\OperatorTok{$}\NormalTok{fit}\OperatorTok{$}\NormalTok{residuals)}\OperatorTok{/}\NormalTok{att_garch_sp500}\OperatorTok{$}\NormalTok{fit}\OperatorTok{$}\NormalTok{var,}
          \DataTypeTok{main =} \StringTok{"Standardized Residuals for GARCH(2,1)"}\NormalTok{)}
\end{Highlighting}
\end{Shaded}

\includegraphics{Project_2_files/figure-latex/unnamed-chunk-9-1.pdf}

From this, we can also see that the standardized squared residuals is
white noise. Thus, we use GARCH(2,1).

\subsubsection{c. Respective Residuals}\label{c.-respective-residuals}

\paragraph{USD/IDR model residuals}\label{usdidr-model-residuals}

\begin{Shaded}
\begin{Highlighting}[]
\KeywordTok{plot}\NormalTok{(att_garch_usdidr}\OperatorTok{$}\NormalTok{fit}\OperatorTok{$}\NormalTok{residuals }\OperatorTok{~}\StringTok{ }\NormalTok{att_garch_usdidr}\OperatorTok{$}\NormalTok{fit}\OperatorTok{$}\NormalTok{fitted.values,}
     \DataTypeTok{main =} \StringTok{"USD/IDR residuals using GARCH(4,1)"}\NormalTok{,}
     \DataTypeTok{xlab =} \StringTok{"Fitted values"}\NormalTok{,}
     \DataTypeTok{ylab =} \StringTok{"Residuals"}\NormalTok{)}
\end{Highlighting}
\end{Shaded}

\includegraphics{Project_2_files/figure-latex/unnamed-chunk-10-1.pdf}

\paragraph{S\&P500 model residuals}\label{sp500-model-residuals}

\begin{Shaded}
\begin{Highlighting}[]
\KeywordTok{plot}\NormalTok{(att_garch_sp500}\OperatorTok{$}\NormalTok{fit}\OperatorTok{$}\NormalTok{residuals }\OperatorTok{~}\StringTok{ }\NormalTok{att_garch_sp500}\OperatorTok{$}\NormalTok{fit}\OperatorTok{$}\NormalTok{fitted.values,}
     \DataTypeTok{main =} \StringTok{"S&P500 residuals using GARCH(2,1)"}\NormalTok{,}
     \DataTypeTok{xlab =} \StringTok{"Fitted values"}\NormalTok{,}
     \DataTypeTok{ylab =} \StringTok{"Residuals"}\NormalTok{)}
\end{Highlighting}
\end{Shaded}

\includegraphics{Project_2_files/figure-latex/unnamed-chunk-11-1.pdf}

The residuals for both data are pretty much randomly distributed. So our
model is considerably good. Next, we can do further analysis on the ACF
and PACF of the residuals.

\subsubsection{e. ACF and PACF of
residuals}\label{e.-acf-and-pacf-of-residuals}

\paragraph{USD/IDR residuals}\label{usdidr-residuals}

\begin{Shaded}
\begin{Highlighting}[]
\KeywordTok{tsdisplay}\NormalTok{(att_garch_usdidr}\OperatorTok{$}\NormalTok{fit}\OperatorTok{$}\NormalTok{residuals)}
\end{Highlighting}
\end{Shaded}

\includegraphics{Project_2_files/figure-latex/unnamed-chunk-12-1.pdf}

There is no significant spikes on both the ACF and PACF which show that
the residuals follows white noise model. We can also check the residuals
using Ljung-Box test with the null defined by:
\[H_{0}:\rho_{1}=\rho_{2}=...=\rho_{k}=0\]

\begin{Shaded}
\begin{Highlighting}[]
\KeywordTok{Box.test}\NormalTok{(att_garch_usdidr}\OperatorTok{$}\NormalTok{fit}\OperatorTok{$}\NormalTok{residuals, }\DataTypeTok{type =} \StringTok{"Ljung-Box"}\NormalTok{)}
\end{Highlighting}
\end{Shaded}

\begin{verbatim}
## 
##  Box-Ljung test
## 
## data:  att_garch_usdidr$fit$residuals
## X-squared = 0.10866, df = 1, p-value = 0.7417
\end{verbatim}

We failed to reject the null hypothesis. So there isn't enough evidence
to say that the residuals are not white noise.

\paragraph{S\&P 500 residuals}\label{sp-500-residuals}

\begin{Shaded}
\begin{Highlighting}[]
\KeywordTok{tsdisplay}\NormalTok{(att_garch_sp500}\OperatorTok{$}\NormalTok{fit}\OperatorTok{$}\NormalTok{residuals)}
\end{Highlighting}
\end{Shaded}

\includegraphics{Project_2_files/figure-latex/unnamed-chunk-14-1.pdf}

Looking at ACF and PACF, it seems there are only single spikes at lag
19. However, it is small so it should be safe to ignore it. For
completeness, we run the Ljung-Box test.

\begin{Shaded}
\begin{Highlighting}[]
\KeywordTok{Box.test}\NormalTok{(att_garch_sp500}\OperatorTok{$}\NormalTok{fit}\OperatorTok{$}\NormalTok{residuals, }\DataTypeTok{type =} \StringTok{"Ljung-Box"}\NormalTok{)}
\end{Highlighting}
\end{Shaded}

\begin{verbatim}
## 
##  Box-Ljung test
## 
## data:  att_garch_sp500$fit$residuals
## X-squared = 7.7541, df = 1, p-value = 0.005359
\end{verbatim}

Using alpha 0.1, we failed to reject the null hypothesis, both the
residuals from our model follow white noise.

\subsubsection{f. CUSUM}\label{f.-cusum}

For testing the parameter stability, we calculate the CUSUM plot. The
red band is the interval of the stability of model.

\paragraph{USD/IDR}\label{usdidr}

\begin{Shaded}
\begin{Highlighting}[]
\KeywordTok{plot}\NormalTok{(}\KeywordTok{efp}\NormalTok{(att_garch_usdidr}\OperatorTok{$}\NormalTok{fit}\OperatorTok{$}\NormalTok{res }\OperatorTok{~}\StringTok{ }\DecValTok{1}\NormalTok{))}
\end{Highlighting}
\end{Shaded}

\includegraphics{Project_2_files/figure-latex/CUSUM usdidr-1.pdf}

\paragraph{S\&P 500}\label{sp-500}

\begin{Shaded}
\begin{Highlighting}[]
\KeywordTok{plot}\NormalTok{(}\KeywordTok{efp}\NormalTok{(att_garch_sp500}\OperatorTok{$}\NormalTok{fit}\OperatorTok{$}\NormalTok{res }\OperatorTok{~}\StringTok{ }\DecValTok{1}\NormalTok{))}
\end{Highlighting}
\end{Shaded}

\includegraphics{Project_2_files/figure-latex/CUSUM sp500-1.pdf}

\subsubsection{g. Recursive Residuals}\label{g.-recursive-residuals}

The other thing to consider in validating the model is to look at its
recursive residuals.

\paragraph{USD/IDR}\label{usdidr-1}

\begin{Shaded}
\begin{Highlighting}[]
\NormalTok{rec_usdidr =}\StringTok{ }\KeywordTok{recresid}\NormalTok{(att_garch_usdidr}\OperatorTok{$}\NormalTok{fit}\OperatorTok{$}\NormalTok{res }\OperatorTok{~}\StringTok{ }\DecValTok{1}\NormalTok{)}
\KeywordTok{plot}\NormalTok{(rec_usdidr, }\DataTypeTok{pch =} \DecValTok{16}\NormalTok{, }\DataTypeTok{main =} \StringTok{"Recursive Residuals"}\NormalTok{, }\DataTypeTok{ylab =} \StringTok{"Recres"}\NormalTok{)}
\end{Highlighting}
\end{Shaded}

\includegraphics{Project_2_files/figure-latex/Recres usdidr-1.pdf}

\paragraph{S\&P 500}\label{sp-500-1}

\begin{Shaded}
\begin{Highlighting}[]
\NormalTok{rec_sp500 =}\StringTok{ }\KeywordTok{recresid}\NormalTok{(att_garch_sp500}\OperatorTok{$}\NormalTok{fit}\OperatorTok{$}\NormalTok{res }\OperatorTok{~}\StringTok{ }\DecValTok{1}\NormalTok{)}
\KeywordTok{plot}\NormalTok{(rec_sp500, }\DataTypeTok{pch =} \DecValTok{16}\NormalTok{, }\DataTypeTok{ylab =} \StringTok{"Recres"}\NormalTok{, }\DataTypeTok{main =} \StringTok{"Recursive Residuals"}\NormalTok{)}
\end{Highlighting}
\end{Shaded}

\includegraphics{Project_2_files/figure-latex/recres sp500-1.pdf}

\subsubsection{h. Diagnostic statistics}\label{h.-diagnostic-statistics}

In conclusion, we derive ARMA (2,2) as our best model. This, in fact is
better than ARMA (2,2) with S-AR (1) which was generated using
auto.arima. This is because the AIC for ARMA (2,2) is smaller and the
coefficient parameter for S-AR(1) is not statistically significant. The
next thing we do is to fit the best GARCH model because the squared
residuals still have spikes in ACF and PACF.

By comparing AIC of all the combination p and q in GARCH (p,q), we get
that GARCH(4,1) is the best for USD/IDR return and GARCH (2,1) is the
best for S\&P 500 returns. Finally, the standardized residuals show no
significant spikes in the plot of ACF and PACF.

Seeing at the CUSUM, we see that the plot of graph for both S\&P500 and
USD/IDR doesn't break the band. Which implies that our model is
consistent. Furthermore, the recursive residuals also tell use that our
model is valid.

\subsubsection{i. Forecast}\label{i.-forecast}

\paragraph{Forecast returns}\label{forecast-returns}

\subparagraph{USD/IDR forecast}\label{usdidr-forecast}

\begin{Shaded}
\begin{Highlighting}[]
\NormalTok{mod_for_usdidr =}\StringTok{ }\KeywordTok{ugarchforecast}\NormalTok{(garchfit41_usdidr, }\DataTypeTok{data =} \OtherTok{NULL}\NormalTok{, }\DataTypeTok{n.ahead =} \DecValTok{12}\NormalTok{, }\DataTypeTok{n.roll =} \DecValTok{0}\NormalTok{, }\DataTypeTok{out.sample =} \DecValTok{0}\NormalTok{)}

\CommentTok{# prediction}
\NormalTok{mod_for_usdidr}
\end{Highlighting}
\end{Shaded}

\begin{verbatim}
## 
## *------------------------------------*
## *       GARCH Model Forecast         *
## *------------------------------------*
## Model: sGARCH
## Horizon: 12
## Roll Steps: 0
## Out of Sample: 0
## 
## 0-roll forecast [T0=May 2019]:
##      Series Sigma
## T+1  0.4060 1.952
## T+2  0.3626 2.252
## T+3  0.2178 2.510
## T+4  0.1722 2.745
## T+5  0.2509 2.962
## T+6  0.3247 3.164
## T+7  0.3079 3.354
## T+8  0.2466 3.533
## T+9  0.2262 3.703
## T+10 0.2589 3.866
## T+11 0.2907 4.022
## T+12 0.2842 4.172
\end{verbatim}

Above is the 12 step-ahead forecast. To see what's those number really
meant, plotting is nessesary.

\begin{Shaded}
\begin{Highlighting}[]
\KeywordTok{plot}\NormalTok{(mod_for_usdidr, }\DataTypeTok{which =} \DecValTok{1}\NormalTok{)}
\end{Highlighting}
\end{Shaded}

\includegraphics{Project_2_files/figure-latex/forecast usdidr plot-1.pdf}

\subparagraph{S\&P 500 forecast}\label{sp-500-forecast}

\begin{Shaded}
\begin{Highlighting}[]
\NormalTok{mod_for_sp500 =}\StringTok{ }\KeywordTok{ugarchforecast}\NormalTok{(garchfit21_sp500, }\DataTypeTok{data =} \OtherTok{NULL}\NormalTok{, }\DataTypeTok{n.ahead =} \DecValTok{12}\NormalTok{, }\DataTypeTok{n.roll =} \DecValTok{0}\NormalTok{, }\DataTypeTok{out.sample =} \DecValTok{0}\NormalTok{)}

\CommentTok{# prediction}
\NormalTok{mod_for_sp500}
\end{Highlighting}
\end{Shaded}

\begin{verbatim}
## 
## *------------------------------------*
## *       GARCH Model Forecast         *
## *------------------------------------*
## Model: sGARCH
## Horizon: 12
## Roll Steps: 0
## Out of Sample: 0
## 
## 0-roll forecast [T0=May 2019]:
##      Series Sigma
## T+1  1.3746 3.606
## T+2  0.9361 3.746
## T+3  0.3347 3.734
## T+4  1.1669 3.751
## T+5  0.7867 3.761
## T+6  0.6045 3.771
## T+7  1.0146 3.780
## T+8  0.7543 3.787
## T+9  0.7315 3.794
## T+10 0.9186 3.801
## T+11 0.7613 3.806
## T+12 0.7863 3.811
\end{verbatim}

\begin{Shaded}
\begin{Highlighting}[]
\CommentTok{# plot}
\KeywordTok{plot}\NormalTok{(mod_for_sp500, }\DataTypeTok{which =} \DecValTok{1}\NormalTok{)}
\end{Highlighting}
\end{Shaded}

\includegraphics{Project_2_files/figure-latex/forecast sp500-1.pdf}

\paragraph{Forecast volatility}\label{forecast-volatility}

Since we're using GARCH model for both series, we can forecast their
volatility. \#\#\#\#\# Volatility for USD/IDR

\begin{Shaded}
\begin{Highlighting}[]
\KeywordTok{plot}\NormalTok{(mod_for_usdidr, }\DataTypeTok{which =} \DecValTok{3}\NormalTok{)}
\end{Highlighting}
\end{Shaded}

\includegraphics{Project_2_files/figure-latex/unnamed-chunk-16-1.pdf}

\subparagraph{Volatility for S\&P 500}\label{volatility-for-sp-500}

\begin{Shaded}
\begin{Highlighting}[]
\KeywordTok{plot}\NormalTok{(mod_for_sp500, }\DataTypeTok{which =} \DecValTok{3}\NormalTok{)}
\end{Highlighting}
\end{Shaded}

\includegraphics{Project_2_files/figure-latex/unnamed-chunk-17-1.pdf}

\subsubsection{i. VAR models}\label{i.-var-models}

Now we are considering VAR models to test our hypothesis about
relationship between data. We first find the best parameter p for VAR(p)
using AIC and BIC.

\begin{Shaded}
\begin{Highlighting}[]
\CommentTok{#VAR}
\NormalTok{varbind =}\StringTok{ }\KeywordTok{cbind}\NormalTok{(sp500_ret, usd_idr_ret)}

\CommentTok{#Choose Best model by AIC BIC}
\NormalTok{row =}\StringTok{ }\DecValTok{20}
\NormalTok{aicbic =}\StringTok{ }\KeywordTok{matrix}\NormalTok{(}\OtherTok{NA}\NormalTok{, }\DataTypeTok{ncol=}\DecValTok{3}\NormalTok{, }\DataTypeTok{nrow=}\NormalTok{row)}
\KeywordTok{colnames}\NormalTok{(aicbic) =}\StringTok{ }\KeywordTok{c}\NormalTok{(}\StringTok{"p"}\NormalTok{, }\StringTok{"AIC"}\NormalTok{, }\StringTok{"BIC"}\NormalTok{)}

\ControlFlowTok{for}\NormalTok{(i }\ControlFlowTok{in} \DecValTok{1}\OperatorTok{:}\NormalTok{row)\{}
\NormalTok{  aicbic[i, }\DecValTok{1}\NormalTok{] =}\StringTok{ }\NormalTok{i}
\NormalTok{  aicbic[i, }\DecValTok{2}\NormalTok{] =}\StringTok{ }\KeywordTok{AIC}\NormalTok{(}\KeywordTok{VAR}\NormalTok{(varbind, }\DataTypeTok{p=}\NormalTok{i))}
\NormalTok{  aicbic[i, }\DecValTok{3}\NormalTok{] =}\StringTok{ }\KeywordTok{BIC}\NormalTok{(}\KeywordTok{VAR}\NormalTok{(varbind, }\DataTypeTok{p=}\NormalTok{i))}
\NormalTok{\}}

\NormalTok{aicbic}
\end{Highlighting}
\end{Shaded}

\begin{verbatim}
##        p      AIC      BIC
##  [1,]  1 2274.074 2294.070
##  [2,]  2 2263.980 2297.259
##  [3,]  3 2249.508 2296.030
##  [4,]  4 2237.086 2296.812
##  [5,]  5 2227.394 2300.285
##  [6,]  6 2217.490 2303.504
##  [7,]  7 2209.034 2308.133
##  [8,]  8 2192.192 2304.334
##  [9,]  9 2177.475 2302.620
## [10,] 10 2173.872 2311.979
## [11,] 11 2169.332 2320.359
## [12,] 12 2164.756 2328.662
## [13,] 13 2159.433 2336.175
## [14,] 14 2157.284 2346.819
## [15,] 15 2148.768 2351.055
## [16,] 16 2143.984 2358.979
## [17,] 17 2139.605 2367.264
## [18,] 18 2129.737 2370.017
## [19,] 19 2119.909 2372.765
## [20,] 20 2113.135 2378.523
\end{verbatim}

\begin{Shaded}
\begin{Highlighting}[]
\KeywordTok{plot}\NormalTok{(aicbic[,}\DecValTok{1}\NormalTok{], aicbic[,}\DecValTok{2}\NormalTok{], }\DataTypeTok{xlab =} \StringTok{"p"}\NormalTok{, }\DataTypeTok{ylab=}\StringTok{"AIC"}\NormalTok{, }\DataTypeTok{main=}\StringTok{"AIC plot"}\NormalTok{)}
\end{Highlighting}
\end{Shaded}

\includegraphics{Project_2_files/figure-latex/unnamed-chunk-18-1.pdf}

\begin{Shaded}
\begin{Highlighting}[]
\KeywordTok{plot}\NormalTok{(aicbic[,}\DecValTok{1}\NormalTok{], aicbic[,}\DecValTok{3}\NormalTok{], }\DataTypeTok{xlab =} \StringTok{"p"}\NormalTok{, }\DataTypeTok{ylab=}\StringTok{"BIC"}\NormalTok{, }\DataTypeTok{main=}\StringTok{"BIC plot"}\NormalTok{)}
\end{Highlighting}
\end{Shaded}

\includegraphics{Project_2_files/figure-latex/unnamed-chunk-18-2.pdf}

From the AIC and BIC plots, we can see that the AIC plots are always
decreasing even after p\textgreater{}20, thus we disregard this. Looking
at the BIC plot, the smallest BIC is when p=1. Therefore, we use VAR(1).

Then, we can make a VAR(1) model,

\begin{Shaded}
\begin{Highlighting}[]
\CommentTok{#Choose VAR(1)}
\NormalTok{varmod =}\StringTok{ }\KeywordTok{VAR}\NormalTok{(varbind, }\DataTypeTok{p=}\DecValTok{1}\NormalTok{)}
\KeywordTok{summary}\NormalTok{(varmod)}
\end{Highlighting}
\end{Shaded}

\begin{verbatim}
## 
## VAR Estimation Results:
## ========================= 
## Endogenous variables: sp500_ret, usd_idr_ret 
## Deterministic variables: const 
## Sample size: 207 
## Log Likelihood: -1131.037 
## Roots of the characteristic polynomial:
## 0.1217 0.0827
## Call:
## VAR(y = varbind, p = 1)
## 
## 
## Estimation results for equation sp500_ret: 
## ========================================== 
## sp500_ret = sp500_ret.l1 + usd_idr_ret.l1 + const 
## 
##                Estimate Std. Error t value Pr(>|t|)
## sp500_ret.l1    0.10436    0.07407   1.409    0.160
## usd_idr_ret.l1  0.03699    0.08145   0.454    0.650
## const           0.47241    0.28682   1.647    0.101
## 
## 
## Residual standard error: 4.065 on 204 degrees of freedom
## Multiple R-Squared: 0.009637,    Adjusted R-squared: -7.214e-05 
## F-statistic: 0.9926 on 2 and 204 DF,  p-value: 0.3724 
## 
## 
## Estimation results for equation usd_idr_ret: 
## ============================================ 
## usd_idr_ret = sp500_ret.l1 + usd_idr_ret.l1 + const 
## 
##                Estimate Std. Error t value Pr(>|t|)  
## sp500_ret.l1   -0.13237    0.06659  -1.988   0.0482 *
## usd_idr_ret.l1 -0.14335    0.07322  -1.958   0.0516 .
## const           0.33932    0.25785   1.316   0.1897  
## ---
## Signif. codes:  0 '***' 0.001 '**' 0.01 '*' 0.05 '.' 0.1 ' ' 1
## 
## 
## Residual standard error: 3.654 on 204 degrees of freedom
## Multiple R-Squared: 0.02781, Adjusted R-squared: 0.01828 
## F-statistic: 2.918 on 2 and 204 DF,  p-value: 0.0563 
## 
## 
## 
## Covariance matrix of residuals:
##             sp500_ret usd_idr_ret
## sp500_ret      16.524      -4.901
## usd_idr_ret    -4.901      13.354
## 
## Correlation matrix of residuals:
##             sp500_ret usd_idr_ret
## sp500_ret      1.0000     -0.3299
## usd_idr_ret   -0.3299      1.0000
\end{verbatim}

Looking at the summary, using USD/IDR Return to explain S\&P500 return,
we can see that the parameters are not significant. Using S\&P500 to
explain USD/IDR, we can only see a little bit of significance at α=0.1.

Then, we look at the Impulse Response Function (IRF) of the datas,

\begin{Shaded}
\begin{Highlighting}[]
\CommentTok{#IRF}
\NormalTok{irf_mod=}\KeywordTok{irf}\NormalTok{(varmod)}
\KeywordTok{plot}\NormalTok{(irf_mod)}
\end{Highlighting}
\end{Shaded}

\includegraphics{Project_2_files/figure-latex/unnamed-chunk-20-1.pdf}
\includegraphics{Project_2_files/figure-latex/unnamed-chunk-20-2.pdf}

In the first graph, we can see that the effect of S\&P500 return shock
on itself has a high effect at first, but decays quickly to zero until
lag=2. The effect of S\&P500 return shock on USD/IDR return has some
effect at first but again, decays to zero quickly.

In the second graph, we can also observe that USD/IDR return shock does
not have any significant effect at all on S\&P500 return. It has a high
effect for itself but decays quickly to zero as well.

\subsubsection{k. Granger-Causality}\label{k.-granger-causality}

To analyze the causal effect of each variable to another, we perform a
Granger-Causality test,

\begin{Shaded}
\begin{Highlighting}[]
\CommentTok{#Granger test 1 (H0 = USD/IDR does not explain SP500)}
\KeywordTok{grangertest}\NormalTok{(sp500_ret }\OperatorTok{~}\StringTok{ }\NormalTok{usd_idr_ret, }\DataTypeTok{order=}\DecValTok{1}\NormalTok{)}
\end{Highlighting}
\end{Shaded}

\begin{verbatim}
## Granger causality test
## 
## Model 1: sp500_ret ~ Lags(sp500_ret, 1:1) + Lags(usd_idr_ret, 1:1)
## Model 2: sp500_ret ~ Lags(sp500_ret, 1:1)
##   Res.Df Df      F Pr(>F)
## 1    204                 
## 2    205 -1 0.2062 0.6503
\end{verbatim}

\begin{Shaded}
\begin{Highlighting}[]
\CommentTok{#Granger test 2 (H0 = SP500 does not explain USD/IDR)}
\KeywordTok{grangertest}\NormalTok{(usd_idr_ret }\OperatorTok{~}\StringTok{ }\NormalTok{sp500_ret, }\DataTypeTok{order=}\DecValTok{1}\NormalTok{)}
\end{Highlighting}
\end{Shaded}

\begin{verbatim}
## Granger causality test
## 
## Model 1: usd_idr_ret ~ Lags(usd_idr_ret, 1:1) + Lags(sp500_ret, 1:1)
## Model 2: usd_idr_ret ~ Lags(usd_idr_ret, 1:1)
##   Res.Df Df      F  Pr(>F)  
## 1    204                    
## 2    205 -1 3.9516 0.04816 *
## ---
## Signif. codes:  0 '***' 0.001 '**' 0.01 '*' 0.05 '.' 0.1 ' ' 1
\end{verbatim}

Seeing both results, the granger test cause does not have any
significant in any H\textsubscript{0}. Although there is some
significant in H\textsubscript{0}: S\&P500 return does not explain
USD/IDR return, it is only at α=0.1 which is not considered strongly
significant.

\subsubsection{l. VAR vs.~ARIMA}\label{l.-var-vs.arima}

Comparing the predictions of VAR and GARCH which includes ARIMA.

\begin{Shaded}
\begin{Highlighting}[]
\CommentTok{#Forecast using VAR}
\KeywordTok{plot}\NormalTok{(}\KeywordTok{forecast}\NormalTok{(varmod, }\DataTypeTok{h=}\DecValTok{12}\NormalTok{))}
\end{Highlighting}
\end{Shaded}

\includegraphics{Project_2_files/figure-latex/unnamed-chunk-22-1.pdf}

\begin{Shaded}
\begin{Highlighting}[]
\CommentTok{#Forecast using GARCH(4,1) and GARCH(2,1) with ARMA(2,2)}
\KeywordTok{plot}\NormalTok{(mod_for_sp500, }\DataTypeTok{which =} \DecValTok{1}\NormalTok{)}
\end{Highlighting}
\end{Shaded}

\includegraphics{Project_2_files/figure-latex/unnamed-chunk-22-2.pdf}

\begin{Shaded}
\begin{Highlighting}[]
\KeywordTok{plot}\NormalTok{(mod_for_usdidr, }\DataTypeTok{which =} \DecValTok{1}\NormalTok{)}
\end{Highlighting}
\end{Shaded}

\includegraphics{Project_2_files/figure-latex/unnamed-chunk-22-3.pdf}

\subsection{III. Conclusion}\label{iii.-conclusion}

Looking at the graphs, we can see more fluctuations in the GARCH
forecast. This may be because the VAR model is not that good as USD/IDR
return and S\&P500 return does not really explain each other. This is
contradictory to our initial prediciton about the returns. S\&P500
return does not explain fully the USD/IDR return because there are still
a lot of macroeconomics tools that we missed in this project.
Furthermore, S\&P500 return itself may not be a perfect representation
of US Gross Domestic Product (GDP) which may be more accurate in
comparing with USD/IDR return.

\subsection{IV. Reference}\label{iv.-reference}

All datas were downloaded from Yahoo Finance.


\end{document}
